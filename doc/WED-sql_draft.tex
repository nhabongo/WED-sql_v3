% O numero máximo de WED-workers que podem ser executados em paralelo, depende
%da quantidade de conecções que o WED-server consegue manter ativas simultaneamente. Caso contrário, um WED-worker pode
%esperar por uma conecção com o WED-server ou abortar a transação.


%--O que é ?

  WED-SQL é uma implementação do paradigma WED-flow que utiliza um sgbd relacional e a linguagem SQL para gerenciar,
definir propriedades e controlar o fluxo de dados de transações longas (Long-lived Transactions). Por utilizar
como base tecnologias consolidadas e amplamente adotadas no universo da computação, a WED-SQL visa difundir a adoçao
do paradigma WED-flow na modelagem de processos de negócio , além de criar uma ferramenta robusta, facilmente
escalável, simples de ser utilizada e que permita flexibilidade na especificação de controle de fluxos.

%--Como funciona ? (wed-server, wed-worker, bg_worker)
%webservice, fault tolerance, keep alive transaction, exception management

  Com o objetivo de  ser utilizada em um ambiente distribuido, mais especificamente por meio de web-services, e também
devido à natureza das transações longas, a WED-SQL foi construida utilizando a arquitetura cliente-servidor. A componente
servidor, ou WED-server, é responsável por fazer o controle de fluxo de dados de acordo com as WED-conditions definidas
para um determinado WED-flow, disparando as WED-triggers conforme necessário. Já a componente cliente, ou WED-worker (
quem sabe até WED-service), é quem efetivamente realiza as WED-transitions.

%postgresl catalog driven, mvcc, extensions
%aproveitar arcabouço transacional, linguagem alto nivel
  O WED-server é, basicamente, uma extensão para o sgbd Postgresql composta de triggers, tabelas de controle e "stored
procedures". A escolha do Postgresql como base do WED-server se deu por diversos motivos, tais quais, ser de código aberto, 
implemetação modular baseada em catálogs (system catalog driven), facilidade de extender suas funcionalidades por meio de
código C e tambem em liguagens de alto nível como Python e Perl. Por ser de código aberto, a licensa do Postgresql garante
que o software possa ser modificado e redistribuido, além de que, o acesso ao código fonte possibilita uma melhor compreensão
dos mecanismos interno, garantido uma implementação mais robusta e eficiente do paradigma WED-flow. O postgresql armazena seus
dados de controle em tabelas que são acessíveis aos usuários, que são chamadas de catálogos de sistema, viabilizando a 
criação de modulos para extender suas funcionalidades. Dados de transações podem ser facilmente obtidos por meio desses 
catálogos de sistema, o que é partircularmente útil no caso do WED-server. Dada a natureza dinâmica do modelo WED-flow,
a utilização de uma linguagem de programação de alto nível, como Python, é fundamental para que a expressividade desse
modelo seja implementada de forma plena. Do ponto de vista de implementação, o WED-server contém código Python, SQL, C e 
aproveita o arcabouço transacional clássico (propriedades ACID, controle de concorrência e etc) fornecido pelo Postgresql.

  Já o WED-worker tem a função de conectar-se ao WED-server para efetuar as WED-transitions. Cada WED-transition é associada
a um ou mais WED-workers, dependendo da demanda de trabalho gerada pelo WED-server, e cada WED-worker é especializado
em realizar uma WED-transition específica. A quantidade máxima de WED-workers trabalhando simultaneamente é limitada
apenas pela quantidade de conecções que um dado WED-server é capaz de manter abertas.
  
  %ESTRUTURA (semantica da aplicacao, workflow)
  Um WED-flow é um conjunto de WED-triggers, que associam WED-conditions, definidas em cima de 
um subconjunto de WED-attributes, e WED-transitions.
  Embora seja possível definir multiplos WED-flows em uma única base de dados, o WED-server permite que cada um deles 
esteja localizado em uma base de dados distinta, de acordo com um determinado significado semântico. Com isso também é 
possível isolar atributos que não devem ser compartilhados entre diferentes WED-flows. 
  
  %ER
  O diagrama ER na figura abaixo representa o modelo de dados gerenciado internamente pelo WED-server. Vale notar que, devido
a flexibilidade de modificaçao da estrutura de dados exigida para representar os WED-states, não é possível capturar todos
os aspectos do paradigma WED-flow por meio de um diagrama Entidade-Relacionamento clássico. Por exemplo,
o relacionamento entre as entidades WED_attr e WED_flow é gerenciado pelo WED-server, ou seja, não há chaves indicando
a relação entre essas duas entidades. A principal razão disso é pertimir adicionar ou remover WED-attributes em tempo de execução, 
o que demanda modificações na estrutura de dados que, por sua vez, podem ser efetuadas de maneira ininterrupta.
  Do relacionamento entre as entidades WED_flow e WED_trig resultam a entidade fraca Job_Pool, que contém as WED-transitions
a serem executadas, e a entidade fraca WED_trace, quem contém os registros de execução de cada instância dos WED-flows.
  ---picture---
  
  %tabelas(wed_flow,wed_attr,wed_trig,wed_trace,job_pool)
  O WED-server utiliza cinco tabelas para definir regras, propriedades e controlar o fluxo de execução dos WED-flows: WED_attr,
WED_flow, WED_trace, WED_trig e Job_pool.
  Os WED-attributes são definidos na tabela wed_attr por um nome e, opcionalmente, por um valor padrão representados 
pelas colunas "aname" e "adv" respectivamente. Cada WED-attribute é identificado univocamente por seu nome que também é 
a chave primária da tabela. Ao criar-se um novo WED-attribute, ou seja, ao inserir-se uma nova linha na tabela wed_attr, 
uma coluna de mesmo nome será automaticamente criada na tabela wed_flow.
  A tabela wed_trig contém as WED-triggers, ou seja, cada linha representa a associação de uma WED-condition com uma 
WED-transition. Possui os seguintes atributos:

  - tgid: identificador único de uma WED-trigger;
  - tgname: atributo opcional que pode ser utilizado para dar um nome à WED-trigger;
  - enabled: permite que a WED-trigger seja desativada;
  - trname: identificador único da WED-transition associada.
  - cname: atributo opcional que pode ser utilizado para dar um nome à WED-condition associada;
  - cpred: predicado da WED-condition associada. Aceita qualquer predicado valido na claúsula WHERE em SQL;
  - cfinal: utilizada para indicar qual é a condição final. Embora apenas uma WED-condition possa ser marcada como final,
           o operado lógico OU pode ser utilizado em seu predicado para definir-se multiplos WED-states finais.
  - timeout: tempo limite para que um WED-worker finalize a WED-transition representada por "trname".

  O histórico de execução das WED-transition fica armazenado na tabela wed_trace, composta dos seguintes atributos:
  
  - wid: referência ao identificador da instancia do WED-flow;
  - state: WED-state, em formato json, que disparou as WED-transitions listadas em "trf";
  - trf: lista de WED-transitions disparadas pelo WED-state em "state";
  - trw: WED-transition que gravou o WED-state representado por "state". Um valor nulo indica o WED-state inicial;
  - status: Indica qual o estado do WED-state em "state". Os possíveis valores são "F","E" ou "R" que indicam que o WED-state
           é final, excessão ou regular, respectivamente.
  - tstmp: indica o momento em que ocorreu o registro. Pode se recuperar a história de execução de uma instância ordenando-se
          as entradas nessa tabela por essa coluna.
  
  A tabela job_pool contém entradas referentes as WED-transitions pendentes que precisam ser executadas pelos WED-workers.
Suas colunas são:
  
  - wid: referência ao identificador da instancia do WED-flow;
  - tgid: referência a WED-trigger que disparou a WED-transition "trname";
  - trname: nome da WED-transition a ser executada;
  - lckid: parâmetro opcional que pode ser utilizado pelos WED-workers para se identificarem. Futuramente, pode ser utilizado
          para fins de autenticação e validação dos WED-workers;
  - timeout: tempo limite para a execução da WED-transition (tempo limite da transação). É uma cópia da coluna de mesmo
            nome da tabela wed_trig;
  - payload: WED-state, em formato json, que disparou a referida WED-transition. Pode ser utilizado, por exemplo, por
            WED-workers que executam WED-transitions associadas a WED-conditions que possuem o operador lógico "OR" em 
            seu predicado.
  
  Finalmente, a tabela wed_flow é o ponto de entrada para inicializar-se uma nova instância. Essa tabela é criada dinâmicamente
de acordo com os WED-attributes definidos na tabela wed_attr. Cada entrada em wed_attr corresponde a uma coluna em wed_flow.
Suas entradas são o WED-state atual de cada instância, ou seja, são tuplas formadas por um identificador, representado
na coluna "wid", e os WED-attributes. 
  

  %FUNCIONAMENTO
  %Definir novo WED-flow (1 por db, atributos de interesse)
  Embora seja possível definir multiplos WED-flows em uma única base de dados, o WED-server permite que cada um deles 
esteja localizado em uma base de dados distinta, de acordo com um determinado significado semântico. Com isso também é 
possível isolar atributos que não devem ser compartilhados entre diferentes WED-flows. 
  Para se criar um novo WED-flow é preciso definir um conjunto de WED-atributes e um conjunto de WED-triggers associando
WED-transitions à WED-conditions. Essa definição, por ora, é expressa em SQL (futuramente em WSQL). Veja um exemplo na
figura <ex1.wsql>
  Note que os valores dos predicados para as WED-conditions, representados por meio da coluna "cpred", tem a mesma sintaxe
utilizada para expressar as restrições de uma cláusula WHERE em SQL. De fato, a expressão em "cpred" será utilizada as-is
pelo WED-server para disparar as WED-transitions. Ao utilizar-se duplo \$ para delimitar essa expressão, elimina-se a necessidade
de "escapar" as aspas simples ou outros caracteres especiais.
  Note também que a condição final de um WED-flow é declarada nessa mesma tabela WED_trig, embora de modo simplificado.
São necessários apenas o predicado em "cpred" e o valor Verdade em "cfinal". Caso não haja uma condição final na tabela
WED_trig, todas as instâncias desse WED-flow terminarao em um WED-state de excessão.
  É recomendado encapsular a definição de um WED-flow em uma única transação, uma vez que, no caso de um erro de sintaxe
na definição de uma WED-trigger, por exemplo, seria necessário remover manualmente do WED-server as definições executadas
até o momento do erro. 
  
  %Definir WED-workers
  Como mencionado anteriormente, quem executa as WED-transitions disparadas pelo WED-server são os WED-workers. Sendo assim,
é necessário criar esses WED-workers e associá-los as respectivas WED-transitions. É recomendado ter ao menos um WED-worker
associado a cada WED-transition.
  Um WED-worker é uma aplicação cliente do WED-server e, por esse motivo, pode ser escrito em qualquer linguagem de progra-
mação que tenha suporte para conectar-se ao sgbd PostgreSQL e que implemente o protocolo de comunicação do WED-server (veja
o capitulo xx). 
  No escopo deste trabalho os WED-workers são escritos na linguagem Python utilizando-se o pacote BaseWorker, que acompanha o WED-SQL.
A figura abaixo ilustra a definição de um novo WED-worker:

  O primeiro passo é importar a classe abstrata BaseClass do pacote BaseWorker. Essa classe implementa a lógica do protocolo
de comunicação com o WED-server e também gerencia as conecções com o mesmo. 
  O proximo passo é de fato criar o WED-worker, definindo uma nova classe concreta que implemente o método wed_trans() e
defina os seguintes atributos:
  - trname: nome da WED-transition que será executada por esse WED-worker. Precisa ser o mesmo utilizado na definicão
           do WED-flow;
  - dbs: parametros da conecção com o WED-server no formato aceito pelo driver psycopg2. No mínimo devem ser especificados
        o usuário, o nome da base de dados onde o WED-flow foi carregado e o nome do WED-worker por meio do campo "application_name";
  - wakeup_interval: é o intervalo de tempo no qual o WED-worker fica suspenso esperando por uma notificação do WED-server (veja capitulo xx);
  
  Esses atributos são utilizados para inicializar a BaseClass.
  O método wed_trans() recebe como parametro o WED-state da instÂncia do WED-flow que disparou a transação, e retorna uma 
string com os novos valores dos WED-attributes dessa mesma instância. A sintaxe utilizada para esse valor de retorno deve
ser a mesma utilizada em uma cláusula SET de uma cláusula UPDATE em SQL. O valor "None" pode ser retornado para abortar
a transação.
  Essa classe concreta que implementa o WED-worker deve então ser instanciada e executada, invocando-se o método run().

%ALGORITMOS (trava update, uma instancia por linha, )

  Após definir-se e carregar-se um WED-flow em uma base de dados do WED-server e inicializar-se seus respectivos WED-workers, uma
nova instância é inicializada inserindo-se um WED-state na tabela WED_flow. Essa nova instância receberá um identificador
único que será utilizado tanto para o registro de sua história de execução, na tabela WED_trace, quanto para o controle
transacional das WED-transitions. Se os valores padrao definidos para os WED-attributes, por meio da coluna "adv", na 
tabela WED_attr forem os valores de um WED-state inicial, basta executar o comando abaixo para se inicializar uma nova
instância:
  INSERT INTO wed_flow DEFAULT VALUES;
  Para cada nova instância é inserida na tabela wed_flow, o WED-server irá comparar os predicados das WED-conditions definidos
na tabela wed_trig com o WED-state representado na nova instância. Para cada condição satisfeita a respectiva WED-transition
será disparada na forma de uma entrada na tabela job_pool e uma notificação será enviada ao WED-worker responsavel
por executá-la. Além disso, serão adicionadas entradas na tabela wed_trace registrando os eventos ocorridos. O WED-server
ficará então aguardando até que uma nova instância ou uma WED-transition seja iniciada. Vale notar que WED-states iniciais 
que não sejam finais e que não disparem ao menos uma WED-transition serão rejeitados pelo WED-server.
  Na sequencia, os WED-workers podem atuar de dois modos distintos: imediatamente ao receber uma notificação do WED-server, ou quando o limite
de tempo de espera por notificações termina (atributo wakeup_interval) e, nesse caso, é preciso consultar a tabela job_pool em
busca de WED-transitions pendentes. Independentemente do modo de atuação e antes de inicializar a transação, cada WED-worker 
precisa solicitar ao WED-server uma trava (advisory_lock) na WED-transition e em qual instância do WED-flow será executada a transação. 
Essa trava é necessária tanto para avisar a outros WED-workers que estejam trabalhando na mesma WED-transition que não a 
executem para a mesma instância, quanto para que o WED-server possa controlar o tempo limite da execução de cada WED-transition. 
Não é possivel executar uma WED-transition sem que o respectivo WED-worker consiga obter essa trava previamente.
  De posse da referida trava, um WED-worker terá que finalizar a transação dentro do tempo limite de execução da WED-transition.
Essa transação é finalizada atualizando-se o WED-state atual da instância em que está sendo executada a WED-transition por meio
de um UPDATE na tabela wed_flow.
  Quando uma instância é atualizada, o WED-server novamente irá verificar se esse novo WED-state dispara novas WED-transitions
e, em caso afirmativo, registrá-las na tabela job_pool, além que registrar os novos eventos em wed_trace. Caso esse novo WED-state
satifaça a condição final e não haja nenhuma WED-transition pendente, essa instancia é marcada como finalizada e não poderá ser
modificada, a menos que sejam inseridos novos WED-attributes ou que sejam modificadas ou adicionadas novas WED-triggers. Em
caso de não houver WED-transitions pendentes e esse WED-state não for final, a instância será marcada como excessão e uma
entrada especial será adicionada a tabela job_pool.

%algoritmo do wed-server
%algoritmo do wed-worker
%ilustração
  
%DETALHES DE FUNCIONAMENTO (excessoes, tras simultaneas,continuous query,protocolo de comunicaocao, balanço: coneccoes x demanda,payload pode nao ser o estado atual)

%escalabilidade (horizontal, vertical, consistencia transacional, pg_shard)
 
%TRABALHOS FUTUROS (linguagem, gerenciador de conecções, remocao de attr, forcar wed-state inicial)





  

 
    
